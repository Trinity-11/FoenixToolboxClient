\section{File System Functions}

% \subsubsection*{Example: C}
% \begin{lstlisting}
%     lorum ipsum
% \end{lstlisting}

% \subsubsection*{Example: Assembler}
% \begin{verbatim}
%     lorum ipsum
% \end{verbatim}

\subsection*{sys\_fsys\_open -- 0xFFE078}
Attempt to open a file in the file system for reading or writing.
Returns a channel number associated with the file.
If the returned number is negative, there was an error opening the file.

The mode parameter indicates how the file should be open and is a bit field, where each bit has a separate meaning:
\begin{itemize}
    \item 0x01: Read
    \item 0x02: Write
    \item 0x04: Create if new
    \item 0x08: Always create
    \item 0x10: Open file if it exists, otherwise create
    \item 0x20: Open file for appending
\end{itemize}

\bigskip		

\begin{tabular}{|l||l|} \hline
Prototype & \lstinline!short sys_fsys_open(const char * path, short mode)! \\ \hline
path & the ASCIIZ string containing the path to the file. \\ \hline
mode & the mode (e.g. r/w/create) \\ \hline
Returns & the channel ID for the open file (negative if error) \\ \hline
\end{tabular}

\subsubsection*{Example: C}
\begin{lstlisting}
    short chan = sys_fsys_open("hello.txt", 0x01);
    if (chan > 0) {
      // File is open for reading
    } else {
      // File was not open... chan has the error number
    }
\end{lstlisting}

\subsubsection*{Example: Assembler}
\begin{verbatim}
    pei #1              ; Push 

    ldx #`path          ; Point to the path
    lda #<>path

    jsl sys_fsys_open   ; Try to open the file

    ply                 ; Clean up the stack

    bit #$ffff          ; Check to see if we opened the file
    bmi error

    ; File is open for reading

error:

    ; There was an error
    ; The error number is in the accumulator

path:
    .null "hello.txt"
\end{verbatim}


\subsection*{sys\_fsys\_close -- 0xFFE07C}
Close a file that was previously opened, given its channel number.
If there were writes done on the channel, those writes will be committed to the block device holding the file.

\bigskip

\begin{tabular}{|l||l|} \hline
Prototype & \lstinline!short sys_fsys_close(short fd)! \\ \hline
fd & the channel ID for the file \\ \hline
Returns & 0 on success, negative number on failure \\ \hline
\end{tabular}

\subsubsection*{Example: C}
\begin{lstlisting}
    short chan = sys_fsys_open(...);
    // ...
    sys_fsys_close(chan);
\end{lstlisting}

\subsubsection*{Example: Assembler}
\begin{verbatim}
    lda chan
    jsl sys_fsys_close
\end{verbatim}


\subsection*{sys\_fsys\_opendir -- 0xFFE080}
Open a directory on a volume for reading, given its path.
Returns a directory handle number on success, or a negative number on failure.

\bigskip

\begin{tabular}{|l||l|} \hline
Prototype & \lstinline!short sys_fsys_opendir(const char * path)! \\ \hline
path & the path to the directory to open \\ \hline
Returns & the handle to the directory if >= 0. An error if < 0 \\ \hline
\end{tabular}

\subsubsection*{Example: C}
\begin{lstlisting}
    short dir = sys_fsys_opendir("/sd0/System");
    if (dir > 0) {
      // dir can be used for reading the directory entries
    } else {
      // There was an error... error number in dir
    }    
\end{lstlisting}

\subsubsection*{Example: Assembler}
\begin{verbatim}
    ldx #`path              ; Point to the path
    lda #<>path

    jsl sys_fsys_opendir    ; Try to open the directory

    bit #$ffff              ; Check to see if we opened the directory
    bmi error

    ; Directory is open for reading

error:

    ; There was an error
    ; The error number is in the accumulator

path:
    .null "/sd0/System"
\end{verbatim}


\subsection*{sys\_fsys\_closedir -- 0xFFE084}
Close a previously open directory, given its number.

\bigskip

\begin{tabular}{|l||l|} \hline
Prototype & \lstinline!short sys_fsys_closedir(short dir)! \\ \hline
dir & the directory handle to close \\ \hline
Returns & 0 on success, negative number on error \\ \hline
\end{tabular}

\subsubsection*{Example: C}
\begin{lstlisting}
    short dir = ... // Number of the directory to close
    sys_fsys_closedir(dir);
\end{lstlisting}

\subsubsection*{Example: Assembler}
\begin{verbatim}
    lda dir                 ; Get the number of the directory to close
    jsl sys_fsys_closedir   ; Close the directory
\end{verbatim}


\subsection*{sys\_fsys\_readdir -- 0xFFE088}
Given the number of an open directory, and a buffer in which to place the data, fetch the file information of the next directory entry.
(See below for details on the \verb+file_info+ structure.)

Returns 0 on success, a negative number on failure.

\bigskip

\begin{tabular}{|l||l|} \hline
Prototype & \lstinline!short sys_fsys_readdir(short dir, p_file_info file)! \\ \hline
dir & the handle of the open directory \\ \hline
file & pointer to the t\_file\_info structure to fill out. \\ \hline
Returns & 0 on success, negative number on failure \\ \hline
\end{tabular}

\subsubsection*{Example: C}
\begin{lstlisting}
    short dir = sys_fsys_opendir("/sd0/System");
    if (dir > 0) {
      // dir can be used for reading the directory entries
      struct s_file_info file;
      if (sys_fsys_readdir(dir, &file_info) == 0) {
        // file_info contains information...
      } else {
        // Could not read the file entry...
      }
    } else {
      // There was an error... error number in dir
    }
\end{lstlisting}

\subsubsection*{Example: Assembler}
\begin{verbatim}
    ldx #`path              ; Point to the path
    lda #<>path

    jsl sys_fsys_opendir    ; Try to open the directory

    bit #$ffff              ; Check to see if we opened the directory
    bmi error

    ; Directory is open for reading

    sta dir                 ; Save the directory number

    pei #`file_info         ; Set the pointer to the file info
    pei #<>file_info

    ; Directory number is already in A

    jsl sys_fsys_readdir    ; Try to read from the directory

    ply                     ; Clean up the stack
    ply

    bit #$ffff              ; If result is <0, there is an error
    bmi error

    ; Entry is loaded into structure at file_info

error:

    ; There was an error
    ; The error number is in the accumulator

path:
    .null "/sd0/System"
file_info:
    .dstruct s_file_info
\end{verbatim}


\subsection*{sys\_fsys\_findfirst -- 0xFFE08C}
Given the path to a directory to search, a search pattern, and a pointer to a \verb+file_info+ structure,
return the first entry in the directory that matches the pattern.

Returns a directory handle on success, a negative number if there is an error

\bigskip

\begin{tabular}{|l||l|} \hline
Prototype & \lstinline!short sys_fsys_findfirst(const char * path, const char * pattern, p_file_info file)! \\ \hline
path & the path to the directory to search \\ \hline
pattern & the file name pattern to search for \\ \hline
file & pointer to the t\_file\_info structure to fill out \\ \hline
Returns & error if negative, otherwise the directory handle to use for subsequent calls \\ \hline
\end{tabular}

\subsubsection*{Example: C}
\begin{lstlisting}
    struct s_file_info file;
    short dir = sys_fsys_findfirst("/hd0/System/", "*.pgx", &file_info);
    if (dir == 0) {
      // file_info contains information...
    } else {
      // Could not read the file entry...
    }
\end{lstlisting}

\subsubsection*{Example: Assembler}
\begin{verbatim}
    pei #`file_info         ; Point to the file_info 
    pei #<>file_info

    pei #`pattern           ; Point to the search pattern
    pei #<>pattern

    ldx #`path              ; Point to the directory to search
    lda #<>path

    jsl sys_fsys_findfirst  ; Try to find the first match

    ply                     ; Clean up the stack
    ply
    ply
    ply

    bit #$ffff              ; Check to see if error (negative)
    bmi error

    ; File info should contain the first match

error:

    ; There was an error

file_info:
    .dstruct s_file_info
pattern:
    .null "*.pgx"
path:
    .null "/sd0/System"
\end{verbatim}


\subsection*{sys\_fsys\_findnext -- 0xFFE090}
Given the directory handle for a previously open search (from \verb+sys_fsys_findfirst+), and a \verb+file_info+ structure,
fill out the structure with the file information of the next file to match the original search pattern.

Returns 0 on success, a negative number if there is an error

\bigskip

\begin{tabular}{|l||l|} \hline
Prototype & \lstinline!short sys_fsys_findnext(short dir, p_file_info file)! \\ \hline
dir & the handle to the directory (returned by fsys\_findfirst) to search \\ \hline
file & pointer to the t\_file\_info structure to fill out \\ \hline
Returns & 0 on success, error if negative \\ \hline
\end{tabular}

\subsubsection*{Example: C}
\begin{lstlisting}
    struct s_file_info file;
    short dir = sys_fsys_findfirst("/hd0/System/", "*.pgx", &file_info);
    if (dir == 0) {
      // file_info contains information...

      // Look for the next...
      short result = sys_fsys_findnext(dir, &file_info);

    } else {
      // Could not read the file entry...
    }
\end{lstlisting}

\subsubsection*{Example: Assembler}
\begin{verbatim}
    pei #`file_info         ; Point to the file_info 
    pei #<>file_info

    pei #`pattern           ; Point to the search pattern
    pei #<>pattern

    ldx #`path              ; Point to the directory to search
    lda #<>path

    jsl sys_fsys_findfirst  ; Try to find the first match

    ply                     ; Clean up the stack
    ply
    ply
    ply

    bit #$ffff              ; Check to see if error (negative)
    bmi error

    sta dir                 ; Save the open directory number

    ; File info should contain the first match

    ; ...

    ; Find the next

    pei #`file_info         ; Point to the file_info
    pei #<>file_info

    lda dir                 ; Get the directory number

    jsl sys_fsys_findnext   ; Try to find the next match

    ply                     ; Clean up the stack
    ply

    bit #$ffff              ; Check to see if error
    bmi error

    ; File info should contain next match

error:

    ; There was an error

file_info:
    .dstruct s_file_info
pattern:
    .null "*.pgx"
path:
    .null "/sd0/System"
\end{verbatim}


\subsection*{sys\_fsys\_get\_label -- 0xFFE094}
Get the label of a volume.

\bigskip

\begin{tabular}{|l||l|} \hline
Prototype & \lstinline!short sys_fsys_get_label(const char * path, char * label)! \\ \hline
path & path to the drive \\ \hline
label & buffer that will hold the label... should be at least 35 bytes \\ \hline
Returns & 0 on success, error if negative \\ \hline
\end{tabular}

\subsubsection*{Example: C}
\begin{lstlisting}
    char label[64];
    short result = sys_fsys_get_label("/sd0", label);
\end{lstlisting}

\subsubsection*{Example: Assembler}
\begin{verbatim}
    pei #`label             ; Point to the label buffer
    pei #<>label

    ldx #`path              ; Point to the path of the drive
    lda #<>path

    jsl sys_fsys_get_label  ; Attempt to get the label

    ply                     ; Clean the stack
    ply

    bit #$ffff              ; Check for an error
    bmi error

    ; We should have the label filled

error:

    ; There was an error

path:
    .null "/sd0"
label:
    .fill 64
\end{verbatim}


\subsection*{sys\_fsys\_set\_label -- 0xFFE098}
Set the label of a volume.

\bigskip

\begin{tabular}{|l||l|} \hline
Prototype & \lstinline!short sys_fsys_set_label(short drive, const char * label)! \\ \hline
drive & drive number \\ \hline
label & buffer that holds the label \\ \hline
Returns & 0 on success, error if negative \\ \hline
\end{tabular}

\subsubsection*{Example: C}
\begin{lstlisting}
    short result = sys_fsys_set_label(0, "FNXSD0");
\end{lstlisting}

\subsubsection*{Example: Assembler}
\begin{verbatim}
    pei #`label             ; Point to the label
    pei #<>label

    lda #0                  ; Set the volume number

    jsl sys_fsys_set_label  ; Attempt to set the label

    ply                     ; Clean the stack
    ply

    bit #$ffff              ; Check for an error
    bmi error

    ; We should have the label updated

error:

    ; There was an error

label:
    .null "FNXSD0"
\end{verbatim}

\subsection*{sys\_fsys\_mkdir -- 0xFFE09C}
Create a directory.

\bigskip

\begin{tabular}{|l||l|} \hline
Prototype & \lstinline!short sys_fsys_mkdir(const char * path)! \\ \hline
path & the path of the directory to create. \\ \hline
Returns & 0 on success, negative number on failure. \\ \hline
\end{tabular}

\subsubsection*{Example: C}
\begin{lstlisting}
    short result = sys_fsys_mkdir("/sd0/Samples");
\end{lstlisting}

\subsubsection*{Example: Assembler}
\begin{verbatim}
    ldx #`path
    lda #<>path

    jsl sys_fsys_mkdir  ; Attempt to create the directory

    bit #$ffff          ; Check for an error
    bmi error

    ; Directory should be created

error:

    ; There was an error

path:
    .null "/sd0/Samples"
\end{verbatim}


\subsection*{sys\_fsys\_delete -- 0xFFE0A0}
Delete a file or directory, given its path. Returns 0 on success, a negative number if there is an error

\bigskip

\begin{tabular}{|l||l|} \hline
Prototype & \lstinline!short sys_fsys_delete(const char * path)! \\ \hline
path & the path of the file or directory to delete. \\ \hline
Returns & 0 on success, negative number on failure. \\ \hline
\end{tabular}

\subsubsection*{Example: C}
\begin{lstlisting}
    short result = sys_fsys_delete("/sd0/test.txt");
\end{lstlisting}

\subsubsection*{Example: Assembler}
\begin{verbatim}
    ldx #`path          ; Point to the path to delete
    lda #<>path

    jsl sys_fsys_delete ; Try to delete the file
    bit #$ffff
    bmi error

    ; File was deleted...

error:

    ; There was an error

path:
    .null "/sd0/test.txt"
\end{verbatim}


\subsection*{sys\_fsys\_rename -- 0xFFE0A4}
Rename a file or directory. Returns 0 on success, a negative number if there is an error

\bigskip

\begin{tabular}{|l||l|} \hline
Prototype & \lstinline!short sys_fsys_rename(const char * old_path, const char * new_path)! \\ \hline
old\_path & he current path to the file \\ \hline
new\_path & the new path for the file \\ \hline
Returns & 0 on success, negative number on failure. \\ \hline
\end{tabular}

\subsubsection*{Example: C}
\begin{lstlisting}
    short result = sys_fsys_rename("/sd0/test.txt", "doc.txt");
\end{lstlisting}

\subsubsection*{Example: Assembler}
\begin{verbatim}
    pei #`new_path      ; Push the pointer to the new name
    pei #<>new_path

    ldx #`old_path      ; Point to the original file name
    lda #<>old_path

    jsl sys_fsys_rename ; Try to rename the file the file

    ply                 ; Clean up the stack
    ply

    bit #$ffff          ; Check for an error
    bmi error

    ; File was named...

error:

    ; There was an error

old_path:
    .null "/sd0/test.txt"
new_path:
    .null "doc.txt"
\end{verbatim}

% \subsection*{sys\_fsys\_set\_cwd -- 0xFFE0A4}
% \begin{tabular}{|l||l|} \hline
% Prototype & \lstinline!short sys_fsys_set_cwd(const char * path)! \\ \hline
% path & the path that should be the new current \\ \hline
% Returns & 0 on success, negative number on failure. \\ \hline
% \end{tabular}

% \subsection*{sys\_fsys\_get\_cwd -- 0xFFE0A8}
% \begin{tabular}{|l||l|} \hline
% Prototype & \lstinline!short sys_fsys_get_cwd(char * path, short size)! \\ \hline
% path & the buffer in which to store the directory \\ \hline
% size & the size of the buffer in bytes \\ \hline
% Returns & 0 on success, negative number on failure. \\ \hline
% \end{tabular}

\subsection*{sys\_fsys\_load -- 0xFFE0B0}
Load a file into memory. This function can either load a file into a specific address provided by the caller,
or to the loading address specified in the file (for executable files). For executable files, the function will also
return the starting address specified in the file.

\bigskip

\begin{tabular}{|l||l|} \hline
Prototype & \lstinline!short sys_fsys_load(const char * path, uint32_t destination, uint32_t * start)! \\ \hline
path & the path to the file to load \\ \hline
destination & the destination address (0 for use file's address) \\ \hline
start & pointer to the long variable to fill with the starting address \\ \hline
Returns & 0 on success, negative number on error \\ \hline
\end{tabular}

\subsubsection*{Example: C}
\begin{lstlisting}
    uint32_t start;
    short result = sys_fsys_load("hello.pgx", 0, &start);
\end{lstlisting}

\subsubsection*{Example: Assembler}
\begin{verbatim}
    pei #`start         ; Push the pointer to the start variable
    pei #<>start

    pei #0              ; Push 0 to leave a load address unspecified
    pei #0

    ldx #`path          ; Point to the file name
    lda #<>path

    jsl sys_fsys_load   ; Try to rename the file the file

    ply                 ; Clean up the stack
    ply
    ply
    ply

    bit #$ffff          ; Check for an error
    bmi error

    ; File was loaded

error:

    ; There was an error

path:
    .null "hello.pgx"
start:
    .dword ?
\end{verbatim}


\subsection*{sys\_fsys\_register\_loader -- 0xFFE0B4}
Register a file loader for a binary file type.
A file loader is a function that takes a channel number for a file to load, a long representing the destination address,
and a pointer to a long for the start address of the program. These last two parameters are the same as are provided the \verb+sys_fsys_load+.

On success, returns 0. If there is an error in registering the loader, returns a negative number.

\bigskip

\begin{tabular}{|l||l|} \hline
Prototype & \lstinline!short sys_fsys_register_loader(const char * extension, p_file_loader loader)! \\ \hline
extension & the file extension to map to \\ \hline
loader & pointer to the file load routine to add \\ \hline
Returns & 0 on success, negative number on error \\ \hline
\end{tabular}

\subsubsection*{Example: C}
\begin{lstlisting}
    short foo_loader(short chan, uint32_t destination, uint32_t * start) {
        // Load file to destination (if provided)
        // If executable, set start to address to run
        return 0; // If successful
     };
     // ...
     short result = sys_fsys_register_loader("FOO", foo_loader);
\end{lstlisting}

\subsection*{sys\_fsys\_stat -- 0xFFE0B8}
Check to see if a file is present. The \verb+s_file_info+ structure will be populated if the file is found.
Returns 0 on success or a negative number on an error.

\bigskip

\begin{tabular}{|l||l|} \hline
Prototype & \lstinline!short sys_fsys_stat(const char * path, p_file_info file)! \\ \hline
path & the path to the file to check \\ \hline
file & pointer to a file info record to fill in, if the file is found. \\ \hline
Returns & 0 on success, negative number on error \\ \hline
\end{tabular}

\subsubsection*{Example: C}
\begin{lstlisting}
    s_file_info file_info;
    short result = sys_fsys_stat("/sd0/fnxboot.pgx", &file_info);
\end{lstlisting}