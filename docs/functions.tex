\chapter{Toolbox Functions}

\section{Calling Convention}

All Toolbox functions are long call functions ({\it i.e.} using the \verb+JSL+ and \verb+RTL+ instructions) using the Calypsi ``simple call'' calling convention:
\begin{itemize}
	\item left-most parameter is placed in the accumulator for 8 and 16-bit types, and the X register and the accumulator for 24 and 32 bit types (X taking the most significant bits).
	\item remaining parameters are pushed on to the stack in right to left order (that is, the second parameter in a call is at the top of the stack just before the \verb+JSL+).
	\item 8-bit types are pushed as 16-bit values to avoid switching register sizes mid-call
	\item 24-bit types are pushed as 32-bit values for the same reason
	\item the return value is placed in the accumulator for 8 and 16-bit types, or in the X register and accumulator for 24 and 32 bit types (most significant bits in the X register).
	\item The caller is responsible for removing the parameters from the stack (if any) after the call returns.
\end{itemize}

Furthermore, Toolbox functions are written so as to save the direct page and data bank registers of the caller and to restore them before returning to the caller. This means that a user program can do whatever it likes with the direct page and data bank registers, and the Toolbox will not interfere with those settings. The Toolbox does use those registers itself, but so long as the user program does not alter the Toolbox's RAM blocks (see the memory maps), there should be no interference between the two.

The Toolbox functions are accessed through a jumptable located in the F256's flash memory, starting at 0xFFE000. Each entry is four bytes long, and the address of each function is called out in their detailed descriptions below.

NOTE: Calypsi's ``simple call'' convention is not the fastest way to pass parameters to functions, and it is not Calypsi's only calling convention. There is also a calling convention that uses pseudo-registers in the direct page to pass parameters. Unfortunately, the rules for which parameter goes where in direct page are rather involved. While that convention is preferable when Calypsi is the only compiler involved, the Toolbox needs to allow for other development tools to be used. The stack based convention is more likely to be supported by other compilers. So speed was traded for broader compatibility.

\section{General Functions}

\subsection*{sys\_proc\_exit -- 0xFFE000}
This function ends the currently running program and returns control to the kernel. It takes a single short argument,
which is the result code that should be passed back to the kernel. This function does not return.

\bigskip

\begin{tabular}{|l|l|} \hline
\multicolumn{2}{|l|}{\lstinline!void sys_proc_exit(short result)!} \\ \hline\hline
result    & the code to return to the kernel \\ \hline
\end{tabular}

\subsubsection*{Example: C}
\begin{lstlisting}
sys_proc_exit(0);     // Quit the program with a result code of 0
\end{lstlisting}

\subsubsection*{Example: Assembler}
\begin{verbatim}
    lda #0                ; Return code of 0
    jsl sys_proc_exit     ; Quit the program
\end{verbatim}

\subsection*{sys\_proc\_run -- 0xFFE0DC}
Load and run an executable binary file.
This function will not return on success, since Foenix Toolbox is single tasking.
Any return value will be an error condition.

\bigskip

\begin{tabular}{|l||l|} \hline
Prototype & \lstinline!short sys_proc_run(const char * path, int argc, char * argv[])! \\ \hline
path & the path to the executable file \\ \hline
argc & the number of arguments passed \\ \hline
argv & the array of string arguments \\ \hline
Returns & the return result of the program \\ \hline
\end{tabular}

\subsubsection*{Example: C}
\begin{lstlisting}
    // Attempt to load and run /sd0/hello.pgx
    // Pass the command name and "test" as the arguments

    int argc = 2;
    char * argv[] = {
       "hello.pgx",
       "test"
    };
    short result = sys_proc_run("/sd0/hello.pgx", argc, argv);
\end{lstlisting}

\subsubsection*{Example: Assembler}
\begin{verbatim}
    pei #`argv          ; Push pointer to the arguments
    pei #<>argv
    pei #2              ; Push the argument count
    ldx #`path          ; Point to the path to load
    lda #<>path
    jsl sys_proc_run    ; Try to load and run the file

    ply                 ; Clean up the stack
    ply
    ply

    ; If we get here, there was an error loading or running
    ; the file. Error number is in the accumulator

    ...

path:
    .null "/sd0/hello.pgx"

argv:
    .null "hello.pgx"
    .null "test"
\end{verbatim}


\subsection*{sys\_get\_info -- 0xFFE020}
Fill out a structure with information about the computer. This information includes the model, the CPU, the amount of memory,
versions of the board and FPGAs, and what optional equipment is installed.
.
There is no return value.

\bigskip

\begin{tabular}{|l||l|} \hline
Prototype & \lstinline!void sys_get_info(p_sys_info info)! \\ \hline
info & pointer to a s\_sys\_info structure to fill out \\ \hline
\end{tabular}

\subsubsection*{Example: C}
\begin{lstlisting}
    struct s_sys_info info;
    sys_get_info(&info);
    printf("Machine: %s\n", info.model_name);	
\end{lstlisting}

\subsubsection*{Example: Assembler}
\begin{verbatim}
    ldx #`info			; Point to the info structure
    lda #<>info
    jsl sys_get_info

    ; The structure at info now has data in it
\end{verbatim}


\subsection*{sys\_mem\_get\_ramtop -- 0xFFE0BC}
Return the limit of accessible system RAM. The address returned is the first byte of memory that user programs may not access.
User programs may use any byte from the bottom of system RAM to RAMTOP - 1.

\bigskip

\begin{tabular}{|l||l|} \hline
Prototype & \lstinline!uint32_t sys_mem_get_ramtop()! \\ \hline
Returns & the address of the first byte of reserved system RAM \\ \hline
\end{tabular}


\subsection*{sys\_mem\_reserve -- 0xFFE0C0}
Reserve a block of memory from the top of system RAM.
This call will reduce the value returned by \lstinline|sys_get_ramtop| and will create a block of memory that user programs and the kernel will not change.
The current user program can load into that memory any code or data it needs to protect after it has quit
(for instance, a terminate-stay-resident code block). \lstinline|sys_mem_reserve| returns the address of the first byte of the block reserved.

NOTE: a reserved block cannot be returned to general use accept by restarting the system.

\bigskip

\begin{tabular}{|l||l|} \hline
Prototype & \lstinline!uint32_t sys_mem_reserve(uint32_t bytes)! \\ \hline
bytes & the number of bytes to reserve \\ \hline
Returns & address of the first byte of the reserved block \\ \hline
\end{tabular}

\subsubsection*{Example: C}
\begin{lstlisting}
    // Reserve a block of 256 bytes...
    uint32_t my_block = sys_mem_reserve(256);    
\end{lstlisting}

\subsubsection*{Example: Assembler}
\begin{verbatim}
    ldx #0                  ; Push the amount requested (256 bytes)
    lda #256
    jsl sys_mem_reserve     ; Attempt to reserve the block
    stx my_block+2          ; Save the address of the block reserved
    sta my_block
\end{verbatim}

\subsection*{sys\_time\_jiffies -- 0xFFE0C4}
Returns the number of ``jiffies'' since system startup.

A jiffy is 1/60 second. This clock counts the number of jiffies since the last system startup, but it is not terribly precise.
This counter should be sufficient for providing timeouts and wait delays on a fairly course level, but it should not be used when precision is required.

\bigskip

\begin{tabular}{|l||l|} \hline
Prototype & \lstinline!uint32_t sys_time_jiffies()! \\ \hline
Returns & the number of jiffies since the last reset \\ \hline
\end{tabular}

\subsubsection*{Example: C}
\begin{lstlisting}
    long jiffies = sys_time_jiffies();
\end{lstlisting}

\subsubsection*{Example: Assembler}
\begin{verbatim}
    jsl sys_time_jiffies    ; Get the time

    ; Jiffy count is now in X:A
\end{verbatim}

\subsection*{sys\_rtc\_set\_time -- 0xFFE0C8}
Sets the date and time in the real time clock. The date and time information is provided in an \verb+s_time+ structure (see below).

\bigskip

\begin{tabular}{|l||l|} \hline
Prototype & \lstinline!void sys_rtc_set_time(p_time time)! \\ \hline
time & pointer to a t\_time record containing the correct time \\ \hline
\end{tabular}

\subsubsection*{Example: C}
\begin{lstlisting}
    struct s_time time;
    
    // time structure is filled in with the current time

    // Set the time in the RTC
    sys_rtc_set_time(&time);
\end{lstlisting}

\subsubsection*{Example: Assembler}
\begin{verbatim}
    ; time structure is filled in with the current time

    ...

    ldx #`time              ; Point to the time structure
    lda #<>time
    jsl sys_rtc_set_time    ; Set the time in the RTC
\end{verbatim}


\subsection*{sys\_rtc\_get\_time -- 0xFFE0CC}
Gets the date and time in the real time clock. The date and time information is provided in an \verb+s_time+ structure (see below).

\bigskip

\begin{tabular}{|l||l|} \hline
Prototype & \lstinline!void sys_rtc_get_time(p_time time)! \\ \hline
time & pointer to a t\_time record in which to put the current time \\ \hline
\end{tabular}

\subsubsection*{Example: C}
\begin{lstlisting}
    struct s_time time;
    // ...
    sys_rtc_get_time(&time);
\end{lstlisting}

\subsubsection*{Example: Assembler}
\begin{verbatim}
    ldx #`time              ; Point to the time structure
    lda #<>time
    jsl sys_rtc_get_time    ; Get the time from the RTC
\end{verbatim}

\subsection*{sys\_kbd\_scancode -- 0xFFE0D0}
Returns the next keyboard scan code (0 if none are available).
Note that reading a scan code directly removes it from being used by the regular console code and may cause some surprising behavior if you combine the two.

See below for details about Foenix scan codes.

\bigskip

\begin{tabular}{|l||l|} \hline
Prototype & \lstinline!uint16_t sys_kbd_scancode()! \\ \hline
Returns & the next scan code from the keyboard... 0 if nothing pending \\ \hline
\end{tabular}

\subsubsection*{Example: C}
\begin{lstlisting}
    // Wait for a keypress
    uint16_t scan_code = 0;
    do {
        // Get the Foenix scan code from the keyboard
        scan_code = sys_kbd_scancode();
    } while (scan_code == 0);
\end{lstlisting}

\subsubsection*{Example: Assembler}
\begin{verbatim}
wait:
    jsl sys_kbd_scancode    ; Get the scan code from the keyboard
    cmp #0                  ; Keep checking until we get a keypress
    beq wait
\end{verbatim}


\subsection*{sys\_kbd\_layout -- 0xFFE0D8}
Sets the keyboard translation tables converting from scan codes to 8-bit character codes.
The table provided is copied by the kernel into its own area of memory, so the memory used in the calling program's memory space may be reused after this call.

Takes a pointer to the new translation tables (see below for details). If this pointer is 0, Foenix Toolbox will reset its translation tables to their defaults.

Returns 0 on success, or a negative number on failure.

\bigskip

\begin{tabular}{|l||l|} \hline
Prototype & \lstinline!short sys_kbd_layout(const char * tables)! \\ \hline
tables & pointer to the keyboard translation tables \\ \hline
Returns & 0 on success, negative number on error \\ \hline
\end{tabular}

\section{Channel Functions}

\subsubsection*{Example: C}
\begin{lstlisting}
\end{lstlisting}

\subsubsection*{Example: Assembler}
\begin{verbatim}
\end{verbatim}


\subsection*{sys\_chan\_read\_b -- 0xFFE024}
Read a single character from the channel. Returns the character, or 0 if none are available.


\bigskip

\begin{tabular}{|l||l|} \hline
Prototype & \lstinline!short sys_chan_read_b(short channel)! \\ \hline
channel & the number of the channel \\ \hline
Returns & the value read (if negative, error) \\ \hline
\end{tabular}

\subsubsection*{Example: C}
\begin{lstlisting}
    // Read a byte from channel #0 (keyboard)
    short b = sys_chan_read_b(0);
    if (b >= 0) {
        // We have valid data from 0-255 in b
    }
\end{lstlisting}

\subsubsection*{Example: Assembler}
\begin{verbatim}
    lda #0                 ; Select channel #0
    jsl sys_chan_read_b    ; Read from the channel
    bit #$ffff             ; If negative...
    bmi error              ; Process an error

    ; We have valid data in A
\end{verbatim}

\subsection*{sys\_chan\_read -- 0xFFE028}
Read bytes from a channel and fill a buffer with them, given the number of the channel and the size of the buffer. Returns the number of bytes read.

\bigskip

\begin{tabular}{|l||l|} \hline
Prototype & \lstinline!short sys_chan_read(short channel, unsigned char * buffer, short size)! \\ \hline
channel & the number of the channel \\ \hline
buffer & the buffer into which to copy the channel data \\ \hline
size & the size of the buffer. \\ \hline
Returns & number of bytes read, any negative number is an error code \\ \hline
\end{tabular}

\subsubsection*{Example: C}
\begin{lstlisting}
    char buffer[80];
    short n = sys_chan_read(0, buffer, 80);
    if (n >= 0) {
        // We correctly read n bytes into the buffer
	} else {
        // We have an error
    }
\end{lstlisting}

\subsubsection*{Example: Assembler}
\begin{verbatim}
    pei #80              ; Push the size of the buffer

    pei #`buffer         ; Push the address of the buffer
    pei #<>buffer
	
    lda #0               ; Select channel #0
	
    jsl sys_chan_read    ; Try to read the bytes from the channel
	
    ply                  ; Clean up the stack
    ply
    ply
	
    bit #$ffff           ; If result is negative...
    bmi error            ; Go to process the error
	
    sta n                ; Otherwise: save the number of bytes read
\end{verbatim}

\subsection*{sys\_chan\_readline -- 0xFFE02C}
\begin{tabular}{|l||l|} \hline
Prototype & \lstinline!short sys_chan_readline(short channel, unsigned char * buffer, short size)! \\ \hline
channel & the number of the channel \\ \hline
buffer & the buffer into which to copy the channel data \\ \hline
size & the size of the buffer \\ \hline
Returns & number of bytes read, any negative number is an error code \\ \hline
\end{tabular}

\subsection*{sys\_chan\_write\_b -- 0xFFE030}
\begin{tabular}{|l||l|} \hline
Prototype & \lstinline!short sys_chan_write_b(short channel, uint8_t b)! \\ \hline
channel & the number of the channel \\ \hline
b & the byte to write \\ \hline
Returns & 0 on success, a negative value on error \\ \hline
\end{tabular}

\subsection*{sys\_chan\_write -- 0xFFE034}
\begin{tabular}{|l||l|} \hline
Prototype & \lstinline!short sys_chan_write(short channel, const uint8_t * buffer, short size)! \\ \hline
channel & the number of the channel \\ \hline
buffer &  \\ \hline
size &  \\ \hline
Returns & number of bytes written, any negative number is an error code \\ \hline
\end{tabular}

\subsection*{sys\_chan\_status -- 0xFFE038}
\begin{tabular}{|l||l|} \hline
Prototype & \lstinline!short sys_chan_status(short channel)! \\ \hline
channel & the number of the channel \\ \hline
Returns & the status of the device \\ \hline
\end{tabular}

\subsection*{sys\_chan\_flush -- 0xFFE03C}
\begin{tabular}{|l||l|} \hline
Prototype & \lstinline!short sys_chan_flush(short channel)! \\ \hline
channel & the number of the channel \\ \hline
Returns & 0 on success, any negative number is an error code \\ \hline
\end{tabular}

\subsection*{sys\_chan\_seek -- 0xFFE040}
\begin{tabular}{|l||l|} \hline
Prototype & \lstinline!short sys_chan_seek(short channel, long position, short base)! \\ \hline
channel & the number of the channel \\ \hline
position & the position of the cursor \\ \hline
base & whether the position is absolute or relative to the current position \\ \hline
Returns & 0 = success, a negative number is an error. \\ \hline
\end{tabular}

\subsection*{sys\_chan\_ioctrl -- 0xFFE044}
\begin{tabular}{|l||l|} \hline
Prototype & \lstinline!short sys_chan_ioctrl(short channel, short command, uint8_t * buffer, short size)! \\ \hline
channel & the number of the channel \\ \hline
command & the number of the command to send \\ \hline
buffer & pointer to bytes of additional data for the command \\ \hline
size & the size of the buffer \\ \hline
Returns & 0 on success, any negative number is an error code \\ \hline
\end{tabular}

\subsection*{sys\_chan\_open -- 0xFFE048}
\begin{tabular}{|l||l|} \hline
Prototype & \lstinline!short sys_chan_open(short dev, const char * path, short mode)! \\ \hline
dev & the device number to have a channel opened \\ \hline
path & a "path" describing how the device is to be open \\ \hline
mode & s the device to be read, written, both? (0x01 = READ flag, 0x02 = WRITE flag, 0x03 = READ and WRITE) \\ \hline
Returns & the number of the channel opened, negative number on error \\ \hline
\end{tabular}

\subsection*{sys\_chan\_close -- 0xFFE04C}
\begin{tabular}{|l||l|} \hline
Prototype & \lstinline!short sys_chan_close(short chan)! \\ \hline
chan & the number of the channel to close \\ \hline
Returns & nothing useful \\ \hline
\end{tabular}

\subsection*{sys\_chan\_swap -- 0xFFE050}
\begin{tabular}{|l||l|} \hline
Prototype & \lstinline!short sys_chan_swap(short channel1, short channel2)! \\ \hline
channel1 & the ID of one of the channels \\ \hline
channel2 & the ID of the other channel \\ \hline
Returns & 0 on success, any other number is an error \\ \hline
\end{tabular}

\subsection*{sys\_chan\_device -- 0xFFE054}
\begin{tabular}{|l||l|} \hline
Prototype & \lstinline!short sys_chan_device(short channel)! \\ \hline
channel & the ID of the channel to query \\ \hline
Returns & the ID of the device associated with the channel, negative number for error \\ \hline
\end{tabular}


\section{Block Device Functions}

\subsection*{\texttt{sys\_bdev\_register}}
\begin{tabular}{|l||l|} \hline
Prototype & \lstinline!short sys_bdev_register(p_dev_block device)! \\ \hline
Address & \texttt{0xFFE05C} \\ \hline
\lstinline!device! & pointer to the description of the device to register \\ \hline
Returns & 0 on succes, negative number on error \\ \hline
\end{tabular}

\subsection*{\texttt{sys\_bdev\_read}}
\begin{tabular}{|l||l|} \hline
Prototype & \lstinline!short sys_bdev_read(short dev, long lba, uint8_t * buffer, short size)! \\ \hline
Address & \texttt{0xFFE060} \\ \hline
\lstinline!dev! & the number of the device \\ \hline
\lstinline!lba! & the logical block address of the block to read \\ \hline
\lstinline!buffer! & the buffer into which to copy the block data \\ \hline
\lstinline!size! & the size of the buffer. \\ \hline
Returns & number of bytes read, any negative number is an error code \\ \hline
\end{tabular}

\subsection*{\texttt{sys\_bdev\_write}}
\begin{tabular}{|l||l|} \hline
Prototype & \lstinline!short sys_bdev_write(short dev, long lba, const uint8_t * buffer, short size)! \\ \hline
Address & \texttt{0xFFE064} \\ \hline
\lstinline!dev! & the number of the device \\ \hline
\lstinline!lba! & the logical block address of the block to write \\ \hline
\lstinline!buffer! & the buffer containing the data to write \\ \hline
\lstinline!size! & the size of the buffer. \\ \hline
Returns & number of bytes written, any negative number is an error code \\ \hline
\end{tabular}

\subsection*{\texttt{sys\_bdev\_status}}
\begin{tabular}{|l||l|} \hline
Prototype & \lstinline!short sys_bdev_status(short dev)! \\ \hline
Address & \texttt{0xFFE068} \\ \hline
\lstinline!dev! & the number of the device \\ \hline
Returns & the status of the device \\ \hline
\end{tabular}

\subsection*{\texttt{sys\_bdev\_flush}}
\begin{tabular}{|l||l|} \hline
Prototype & \lstinline!short sys_bdev_flush(short dev)! \\ \hline
Address & \texttt{0xFFE06C} \\ \hline
\lstinline!dev! & the number of the device \\ \hline
Returns & 0 on success, any negative number is an error code \\ \hline
\end{tabular}

\subsection*{\texttt{sys\_bdev\_ioctrl}}
\begin{tabular}{|l||l|} \hline
Prototype & \lstinline!short sys_bdev_ioctrl(short dev, short command, uint8_t * buffer, short size)! \\ \hline
Address & \texttt{0xFFE070} \\ \hline
\lstinline!dev! & the number of the device \\ \hline
\lstinline!command! & the number of the command to send \\ \hline
\lstinline!buffer! & pointer to bytes of additional data for the command \\ \hline
\lstinline!size! & the size of the buffer \\ \hline
Returns & 0 on success, any negative number is an error code \\ \hline
\end{tabular}

\section{File System Functions}

\subsection*{\texttt{sys\_fsys\_open}}
\begin{tabular}{|l||l|} \hline
Prototype & \lstinline!short sys_fsys_open(const char * path, short mode)! \\ \hline
Address & \texttt{0xFFE074} \\ \hline
\lstinline!path! & the ASCIIZ string containing the path to the file. \\ \hline
\lstinline!mode! & the mode (e.g. r/w/create) \\ \hline
Returns & the channel ID for the open file (negative if error) \\ \hline
\end{tabular}

\subsection*{\texttt{sys\_fsys\_close}}
\begin{tabular}{|l||l|} \hline
Prototype & \lstinline!short sys_fsys_close(short fd)! \\ \hline
Address & \texttt{0xFFE078} \\ \hline
\lstinline!fd! & the channel ID for the file \\ \hline
Returns & 0 on success, negative number on failure \\ \hline
\end{tabular}

\subsection*{\texttt{sys\_fsys\_opendir}}
\begin{tabular}{|l||l|} \hline
Prototype & \lstinline!short sys_fsys_opendir(const char * path)! \\ \hline
Address & \texttt{0xFFE07C} \\ \hline
\lstinline!path! & the path to the directory to open \\ \hline
Returns & the handle to the directory if >= 0. An error if < 0 \\ \hline
\end{tabular}

\subsection*{\texttt{sys\_fsys\_closedir}}
\begin{tabular}{|l||l|} \hline
Prototype & \lstinline!short sys_fsys_closedir(short dir)! \\ \hline
Address & \texttt{0xFFE080} \\ \hline
\lstinline!dir! & the directory handle to close \\ \hline
Returns & 0 on success, negative number on error \\ \hline
\end{tabular}

\subsection*{\texttt{sys\_fsys\_readdir}}
\begin{tabular}{|l||l|} \hline
Prototype & \lstinline!short sys_fsys_readdir(short dir, p_file_info file)! \\ \hline
Address & \texttt{0xFFE084} \\ \hline
\lstinline!dir! & the handle of the open directory \\ \hline
\lstinline!file! & pointer to the t\_file\_info structure to fill out. \\ \hline
Returns & 0 on success, negative number on failure \\ \hline
\end{tabular}

\subsection*{\texttt{sys\_fsys\_findfirst}}
\begin{tabular}{|l||l|} \hline
Prototype & \lstinline!short sys_fsys_findfirst(const char * path, const char * pattern, p_file_info file)! \\ \hline
Address & \texttt{0xFFE088} \\ \hline
\lstinline!path! & the path to the directory to search \\ \hline
\lstinline!pattern! & the file name pattern to search for \\ \hline
\lstinline!file! & pointer to the t\_file\_info structure to fill out \\ \hline
Returns & the directory handle to use for subsequent calls if >= 0, error if negative \\ \hline
\end{tabular}

\subsection*{\texttt{sys\_fsys\_findnext}}
\begin{tabular}{|l||l|} \hline
Prototype & \lstinline!short sys_fsys_findnext(short dir, p_file_info file)! \\ \hline
Address & \texttt{0xFFE08C} \\ \hline
\lstinline!dir! & the handle to the directory (returned by fsys\_findfirst) to search \\ \hline
\lstinline!file! & pointer to the t\_file\_info structure to fill out \\ \hline
Returns & 0 on success, error if negative \\ \hline
\end{tabular}

\subsection*{\texttt{sys\_fsys\_get\_label}}
\begin{tabular}{|l||l|} \hline
Prototype & \lstinline!short sys_fsys_get_label(const char * path, char * label)! \\ \hline
Address & \texttt{0xFFE090} \\ \hline
\lstinline!path! & path to the drive \\ \hline
\lstinline!label! & buffer that will hold the label... should be at least 35 bytes \\ \hline
Returns & 0 on success, error if negative \\ \hline
\end{tabular}

\subsection*{\texttt{sys\_fsys\_set\_label}}
\begin{tabular}{|l||l|} \hline
Prototype & \lstinline!short sys_fsys_set_label(short drive, const char * label)! \\ \hline
Address & \texttt{0xFFE094} \\ \hline
\lstinline!drive! & drive number \\ \hline
\lstinline!label! & buffer that holds the label \\ \hline
Returns & 0 on success, error if negative \\ \hline
\end{tabular}

\subsection*{\texttt{sys\_fsys\_mkdir}}
\begin{tabular}{|l||l|} \hline
Prototype & \lstinline!short sys_fsys_mkdir(const char * path)! \\ \hline
Address & \texttt{0xFFE098} \\ \hline
\lstinline!path! & the path of the directory to create. \\ \hline
Returns & 0 on success, negative number on failure. \\ \hline
\end{tabular}

\subsection*{\texttt{sys\_fsys\_delete}}
\begin{tabular}{|l||l|} \hline
Prototype & \lstinline!short sys_fsys_delete(const char * path)! \\ \hline
Address & \texttt{0xFFE09C} \\ \hline
\lstinline!path! & the path of the file or directory to delete. \\ \hline
Returns & 0 on success, negative number on failure. \\ \hline
\end{tabular}

\subsection*{\texttt{sys\_fsys\_rename}}
\begin{tabular}{|l||l|} \hline
Prototype & \lstinline!short sys_fsys_rename(const char * old_path, const char * new_path)! \\ \hline
Address & \texttt{0xFFE0A0} \\ \hline
\lstinline!old_path! & he current path to the file \\ \hline
\lstinline!new_path! & the new path for the file \\ \hline
Returns & 0 on success, negative number on failure. \\ \hline
\end{tabular}

\subsection*{\texttt{sys\_fsys\_set\_cwd}}
\begin{tabular}{|l||l|} \hline
Prototype & \lstinline!short sys_fsys_set_cwd(const char * path)! \\ \hline
Address & \texttt{0xFFE0A4} \\ \hline
\lstinline!path! & the path that should be the new current \\ \hline
Returns & 0 on success, negative number on failure. \\ \hline
\end{tabular}

\subsection*{\texttt{sys\_fsys\_get\_cwd}}
\begin{tabular}{|l||l|} \hline
Prototype & \lstinline!short sys_fsys_get_cwd(char * path, short size)! \\ \hline
Address & \texttt{0xFFE0A8} \\ \hline
\lstinline!path! & the buffer in which to store the directory \\ \hline
\lstinline!size! & the size of the buffer in bytes \\ \hline
Returns & 0 on success, negative number on failure. \\ \hline
\end{tabular}

\subsection*{\texttt{sys\_fsys\_load}}
\begin{tabular}{|l||l|} \hline
Prototype & \lstinline!short sys_fsys_load(const char * path, uint32_t destination, uint32_t * start)! \\ \hline
Address & \texttt{0xFFE0AC} \\ \hline
\lstinline!path! & the path to the file to load \\ \hline
\lstinline!destination! & the destination address (0 for use file's address) \\ \hline
\lstinline!start! & pointer to the long variable to fill with the starting address \\ \hline
Returns & 0 on success, negative number on error \\ \hline
\end{tabular}

\subsection*{\texttt{sys\_fsys\_register\_loader}}
\begin{tabular}{|l||l|} \hline
Prototype & \lstinline!short sys_fsys_register_loader(const char * extension, p_file_loader loader)! \\ \hline
Address & \texttt{0xFFE0B0} \\ \hline
\lstinline!extension! & the file extension to map to \\ \hline
\lstinline!loader! & pointer to the file load routine to add \\ \hline
Returns & 0 on success, negative number on error \\ \hline
\end{tabular}

\subsection*{\texttt{sys\_fsys\_stat}}
\begin{tabular}{|l||l|} \hline
Prototype & \lstinline!short sys_fsys_stat(const char * path, p_file_info file)! \\ \hline
Address & \texttt{0xFFE0B4} \\ \hline
\lstinline!path! & the path to the file to check \\ \hline
\lstinline!file! & pointer to a file info record to fill in, if the file is found. \\ \hline
Returns & 0 on success, negative number on error \\ \hline
\end{tabular}

\section{Text System Functions}

\subsection*{sys\_txt\_set\_mode -- 0xFFE0E0}
\begin{tabular}{|l||l|} \hline
Prototype & \lstinline!short sys_txt_set_mode(short screen, short mode)! \\ \hline
screen & the number of the text device \\ \hline
mode & a bitfield of desired display mode options \\ \hline
Returns & 0 on success, any other number means the mode is invalid for the screen \\ \hline
\end{tabular}

\subsection*{sys\_txt\_set\_xy -- 0xFFE0E8}
\begin{tabular}{|l||l|} \hline
Prototype & \lstinline!void sys_txt_set_xy(short screen, short x, short y)! \\ \hline
screen & the number of the text device \\ \hline
x & the column for the cursor \\ \hline
y & the row for the cursor \\ \hline
\end{tabular}

\subsection*{sys\_txt\_get\_xy -- 0xFFE0EC}
\begin{tabular}{|l||l|} \hline
Prototype & \lstinline!void sys_txt_get_xy(short screen, p_point position)! \\ \hline
screen & the number of the text device \\ \hline
position & pointer to a t\_point record to fill out \\ \hline
\end{tabular}

\subsection*{sys\_txt\_get\_region -- 0xFFE0F0}
\begin{tabular}{|l||l|} \hline
Prototype & \lstinline!short sys_txt_get_region(short screen, p_rect region)! \\ \hline
screen & the number of the text device \\ \hline
region & pointer to a t\_rect describing the rectangular region (using character cells for size and size) \\ \hline
Returns & 0 on success, any other number means the region was invalid \\ \hline
\end{tabular}

\subsection*{sys\_txt\_set\_region -- 0xFFE0F4}
\begin{tabular}{|l||l|} \hline
Prototype & \lstinline!short sys_txt_set_region(short screen, p_rect region)! \\ \hline
screen & the number of the text device \\ \hline
region & pointer to a t\_rect describing the rectangular region (using character cells for size and size) \\ \hline
Returns & 0 on success, any other number means the region was invalid \\ \hline
\end{tabular}

\subsection*{sys\_txt\_set\_color -- 0xFFE0F8}
\begin{tabular}{|l||l|} \hline
Prototype & \lstinline!void sys_txt_set_color(short screen, unsigned char foreground, unsigned char background)! \\ \hline
screen & the number of the text device \\ \hline
foreground & the Text LUT index of the new current foreground color (0 - 15) \\ \hline
background & the Text LUT index of the new current background color (0 - 15) \\ \hline
\end{tabular}

\subsection*{sys\_txt\_get\_color -- 0xFFE0FC}
\begin{tabular}{|l||l|} \hline
Prototype & \lstinline!void sys_txt_get_color(short screen, unsigned char * foreground, unsigned char * background)! \\ \hline
screen & the number of the text device \\ \hline
foreground & the Text LUT index of the new current foreground color (0 - 15) \\ \hline
background & the Text LUT index of the new current background color (0 - 15) \\ \hline
\end{tabular}

\subsection*{sys\_txt\_set\_cursor\_visible -- 0xFFE100}
\begin{tabular}{|l||l|} \hline
Prototype & \lstinline!void sys_txt_set_cursor_visible(short screen, short is_visible)! \\ \hline
screen & the screen number 0 for channel A, 1 for channel B \\ \hline
is\_visible & TRUE if the cursor should be visible, FALSE (0) otherwise \\ \hline
\end{tabular}

\subsection*{sys\_txt\_set\_font -- 0xFFE104}
\begin{tabular}{|l||l|} \hline
Prototype & \lstinline!short sys_txt_set_font(short screen, short width, short height, unsigned char * data)! \\ \hline
screen & the number of the text device \\ \hline
width & width of a character in pixels \\ \hline
height & of a character in pixels \\ \hline
data & pointer to the raw font data to be loaded \\ \hline
\end{tabular}

\subsection*{sys\_txt\_setsizes -- 0xFFE0E4}
\begin{tabular}{|l||l|} \hline
Prototype & \lstinline!void sys_txt_setsizes(short chan)! \\ \hline
chan &  \\ \hline
\end{tabular}

\subsection*{sys\_txt\_get\_sizes -- 0xFFE108}
\begin{tabular}{|l||l|} \hline
Prototype & \lstinline!void sys_txt_get_sizes(short screen, p_extent text_size, p_extent pixel_size)! \\ \hline
screen & the screen number 0 for channel A, 1 for channel B \\ \hline
text\_size & the size of the screen in visible characters (may be null) \\ \hline
pixel\_size & the size of the screen in pixels (may be null) \\ \hline
\end{tabular}

\subsection*{sys\_txt\_set\_border -- 0xFFE10C}
\begin{tabular}{|l||l|} \hline
Prototype & \lstinline!void sys_txt_set_border(short screen, short width, short height)! \\ \hline
screen & the number of the text device \\ \hline
width & the horizontal size of one side of the border (0 - 32 pixels) \\ \hline
height & the vertical size of one side of the border (0 - 32 pixels) \\ \hline
\end{tabular}

\subsection*{sys\_txt\_set\_border\_color -- 0xFFE110}
\begin{tabular}{|l||l|} \hline
Prototype & \lstinline!void sys_txt_set_border_color(short screen, unsigned char red, unsigned char green, unsigned char blue)! \\ \hline
screen & the number of the text device \\ \hline
red & the red component of the color (0 - 255) \\ \hline
green & the green component of the color (0 - 255) \\ \hline
blue & the blue component of the color (0 - 255) \\ \hline
\end{tabular}

\subsection*{sys\_txt\_put -- 0xFFE114}
\begin{tabular}{|l||l|} \hline
Prototype & \lstinline!void sys_txt_put(short screen, char c)! \\ \hline
screen & the number of the text device \\ \hline
c & the character to print \\ \hline
\end{tabular}

\subsection*{sys\_txt\_print -- 0xFFE118}
\begin{tabular}{|l||l|} \hline
Prototype & \lstinline!void sys_txt_print(short screen, const char * message)! \\ \hline
screen & the number of the text device \\ \hline
message & the ASCII Z string to print \\ \hline
\end{tabular}

\section{Interrupt Functions}

\subsection*{sys\_int\_enable\_all -- 0xFFE004}
This function enables all maskable interrupts at the CPU level. It returns a system-dependent code that represents the previous level of interrupt masking. Note: this does not change the mask status of interrupts in the machine's interrupt controller, it just changes if the CPU ignores IRQs or not.

\bigskip

\begin{tabular}{|l||l|} \hline
Prototype & \lstinline!void sys_int_enable_all()! \\ \hline
\end{tabular}

\subsubsection*{Example: C}
\begin{lstlisting}
// Enable processing of IRQs
sys_int_enable_all();
\end{lstlisting}

\subsubsection*{Example: Assembler}
\begin{verbatim}
; Enable processing of IRQs
jrl sys_int_enable_all
\end{verbatim}

\subsection*{sys\_int\_disable\_all -- 0xFFE008}
This function disables all maskable interrupts at the CPU level. It returns a system-dependent code that represents the previous level of interrupt masking. Note: this does not change the mask status of interrupts in the machine's interrupt controller, it just changes if the CPU ignores IRQs or not.

\bigskip

\begin{tabular}{|l||l|} \hline
Prototype & \lstinline!void sys_int_disable_all()! \\ \hline
\end{tabular}

\subsubsection*{Example: C}
\begin{lstlisting}
// Disable processing of IRQs
sys_int_disable_all();
\end{lstlisting}

\subsubsection*{Example: Assembler}
\begin{verbatim}
; Disable processing of IRQs
jrl sys_int_disable_all
\end{verbatim}

\subsection*{sys\_int\_disable -- 0xFFE00C}
This function disables a particular interrupt at the level of the interrupt controller. The argument passed is the number of the interrupt to disable.

\bigskip

\begin{tabular}{|l||l|} \hline
Prototype & \lstinline!void sys_int_disable(unsigned short n)! \\ \hline
n & the number of the interrupt: n[7..4] = group number, n[3..0] = individual number. \\ \hline
\end{tabular}

\subsubsection*{Example: C}
\begin{lstlisting}
// Disable the start-of-frame interrupt
sys_int_disable(INT_SOF_A);
\end{lstlisting}

\subsubsection*{Example: Assembler}
\begin{verbatim}
    lda #INT_SOF_A        ; Enable the start-of-frame interrupt
    jsl sys_int_disable
\end{verbatim}

\subsection*{sys\_int\_enable -- 0xFFE010}
This function enables a particular interrupt at the level of the interrupt controller. The argument passed is the number of the interrupt to enable. Note that interrupts that are enabled at this level will still be disabled, if interrupts are disabled globally by \verb+sys_int_disable_all+.

\bigskip

\begin{tabular}{|l||l|} \hline
Prototype & \lstinline!void sys_int_enable(unsigned short n)! \\ \hline
n & the number of the interrupt \\ \hline
\end{tabular}

\subsubsection*{Example: C}
\begin{lstlisting}
// Enable the start-of-frame interrupt
sys_int_enable(INT_SOF_A);
\end{lstlisting}

\subsubsection*{Example: Assembler}
\begin{verbatim}
    lda #INT_SOF_A        ; Enable the start-of-frame interrupt
    jsl sys_int_enable
\end{verbatim}

\subsection*{sys\_int\_register -- 0xFFE014}
Registers a function as an interrupt handler. An interrupt handler is a function which takes and returns no arguments and will be run in at an elevated privilege level during the interrupt handling cycle.

The first argument is the number of the interrupt to handle, the second argument is a pointer to the interrupt handler to register. Registering a null pointer as an interrupt handler will “deregister” the old handler.

The function returns the handler that was previously registered.

\begin{tabular}{|l||l|} \hline
Prototype & \lstinline!p_int_handler sys_int_register(unsigned short n, p_int_handler handler)! \\ \hline
n & the number of the interrupt \\ \hline
handler & pointer to the interrupt handler to register \\ \hline
Returns & the pointer to the previous interrupt handler \\ \hline
\end{tabular}

\subsubsection*{Example: C}
\begin{lstlisting}
// Handler for the start-of-frame interrupt
// Must be a far sub-routine (returns through RTL)
__attribute__((far)) void sof_handler() {
	// Interrupt handler code here...
}

// Register a handler for the start-of-frame interrupt
p_int_handler old = sys_int_register(INT_SOF_A, sof_handler);
\end{lstlisting}

\subsubsection*{Example: Assember}
\begin{verbatim}
; Handler for the start-of-frame interrupt
; Must be a far sub-routine (returns through RTL)
sof_handler:
    ; Handler code here...
    rtl

    ; Code to register the handler...
    pei #`sof_handler       ; push pointer to sof_handler
    pei #<>sof_handler

    lda #INT_SOF_A          ; A = the number for the SOF_A interrupt
	
    jsl sys_int_register

    sta old                 ; Save the pointer to the old handler
    stx old+2
\end{verbatim}

\subsection*{sys\_int\_pending -- 0xFFE018}
Query an interrupt to see if it is pending in the interrupt controller.
NOTE: User programs will probably never need to use this call, since it is handled by the Toolbox itself.

\bigskip

\begin{tabular}{|l||l|} \hline
Prototype & \lstinline!short sys_int_pending(unsigned short n)! \\ \hline
n & the number of the interrupt: n[7..4] = group number, n[3..0] = individual number. \\ \hline
Returns & non-zero if interrupt n is pending, 0 if not \\ \hline
\end{tabular}

\subsubsection*{Example: C}
\begin{lstlisting}
// Check to see if start-of-frame interrupt is pending
short is_pending = sys_int_pending(INT_SOF_A);
if (is_pending) {
	// The interrupt has not yet been acknowledged
}
\end{lstlisting}

\subsubsection*{Example: Assembler}
\begin{verbatim}
    ; Check to see if the start-of-frame interrupt is pending
    lda #INT_SOF_A
    jsl sys_int_pending
    cmp #0
    beq sof_not_pending

    ; Code for when start-of-frame is pending

sof_not_pending:
\end{verbatim}

\subsection*{sys\_int\_clear -- 0xFFE020}
This function acknowledges the processing of an interrupt by clearing its pending flag in the interrupt controller.
NOTE: User programs will probably never need to use this call, since it is handled by the Toolbox itself.

\bigskip

\begin{tabular}{|l||l|} \hline
Prototype & \lstinline!void sys_int_clear(unsigned short n)! \\ \hline
n & the number of the interrupt: n[7..4] = group number, n[3..0] = individual number. \\ \hline
\end{tabular}

\subsubsection*{Example: C}
\begin{lstlisting}
// Acknowledge the processing of the start-of-frame interrupt
sys_int_clear(INT_SOF_A);
\end{lstlisting}

\subsubsection*{Example: Assembler}
\begin{verbatim}
    ; Acknowledge the processing of the start-of-frame interrupt
    lda #INT_SOF_A
    jsl sys_int_clear
\end{verbatim}